\chapter{Ferramentas e funcionalidades}

\section{Ferramenta de respostas rápidas para perguntas frequentes: FAQ - \textit{Frequently Asked Questions}}

	Para a solução de FAQ, compreendeu-se a necessidade de desenvolvimento das ferramentas:
	
	\begin{enumerate}
		
		\item 
		Modelo de árvore de decisão
		
		Realização de consultar em banco de dados relacional baseado em SQL - \textit{Server Query Language}, o qual localiza uma resposta de solução ao problema alegado.
		
		\item 
		Tela inicial
		
		Implementação de ferramenta de busca para agregar na consulta otimizada afim de facilitar a procura de problemas relacionados.
		
		\item 
		Adicionar Cookie de sessão para armazenar o comportamento do usuário. (verificar IP - \textit{Internet Protocol} - público e contexto LGPD - Lei Geral de Proteção de Dados)
		
		Armazenamento do IP público do usuário para identificar o seu comportamento no site, seguindo critérios de aceite aos termos e políticas condicionais no site, baseadas na LGPD.
		
		\item 
		Cadastro no FAQ
		
		O cadastro no FAQ é realizado pelo próprio time da central de suporte técnico, quando identificam um novo problema que está sendo relatado com muita frequência, busca uma solução otimizada e disponibiliza no FAQ. 
		
	\end{enumerate}

\section{Login}

	\begin{enumerate}
		
		\item 
		Login confiável por autorização de acesso
		
	\end{enumerate}

\section{Cadastros}

	\begin{enumerate}
		
		\item 
		Criação de cadastro de perfil
		
		Após a realização do cadastro com o mínimo necessários de informação o cliente poderá fazer um preenchimento complementar do seu perfil, demonstrando:
		
			\begin{enumerate}
				
				\item 
				Marcas e modelos de seus equipamentos;
				
				\item
				Quantidade de usuários e nome dos usuários no local;
				
				\item
				Software de que gosta de utilizar;
				
				\item
				Outras opções.
				
			\end{enumerate}
		
		\item 
		Criação de cadastro de perfil técnico
		
		Após a realização do cadastro com o mínimo necessários de informação o cliente poderá fazer um preenchimento complementar do seu perfil, demonstrando
		
			\begin{enumerate}
				
				\item
				Marcas e modelos de seus equipamentos que atende;
				
				\item
				Formação profissional;
				
				\item
				Especialização;
				
				\item
				Área que realizará o atendimento de preferência;
				
				\item
				Outras opções. 
				
			\end{enumerate}

	\end{enumerate}

\section{Visualização do perfil e escolha personalizada do técnico}

	\begin{enumerate}
		
		\item
		Visualização de cadastro de perfil
		
		O cliente poderá receber o perfil do técnico e sua média de avaliação e suas especialidades.
		
		\item
		Escolha de técnicos personalizada por critérios de avaliação
		
		O cliente poderá por meio de um plano especifico contratar um técnico com um perfil adequado a sua necessidade e baseado em sua avaliação.
		
	\end{enumerate}

\section{Agendamento de visita técnica}

	O cliente poderá por meio de um plano especifico contratar um técnico com um perfil adequado a sua necessidade e baseado em sua avaliação.

	\begin{enumerate}
	
	     \item
	     Ter opção de botão dedicado para o usuário final abrir um chamado de visita técnica on site.
	     
	     Opção de chamado técnico facilitado, o cliente com um cadastro simples, sem necessidade de acessar o FAQ poderá solicitar um técnico até o local, sendo guiado pelo processo de agendamento.
	     
	     \item
	     Agendamento por meio de raio de localidade
	     
	     Durante o processo de agendamento a escolha do técnico e feita é realizado por localidade do técnico registrado em uma determinada região.
	        
	\end{enumerate}

\section{Operacional}

	\begin{enumerate}
		
		\item
		Abertura do chamado técnico
		
		Todos os atendimentos técnicos realizados pela central de atendimento ou diretamente na visita técnica vão gerar uma abertura de um chamado técnico (incidente).
		
		\item
		Envio de foto do problema do equipamento
		
		O chamado possui campo para adicionar fotos do problema técnico alegado pelo cliente para armazenamento de histórico.
		
	\end{enumerate}

\section{Planos de Compra}

	\begin{enumerate}
		
		\item
		Free
		
		Usuário acessa a plataforma, navega pelos FAQs, podendo sanar suas dúvidas e resolver seus problemas por conta própria através da plataforma – Contém ADS.
		
		\item
		Basic
		
		Modelo de assinatura mensal onde o usuário paga um valor e terá 1 dispositivo vinculado à assinatura. Ao assinar, a equipe instalará os softwares para acesso remoto e, quando o usuário não conseguir resolver por conta própria o problema, será atendido via chat ou WhatsApp para resolução. Para atendimento técnico, os técnicos são escolhidos de forma aleatória dando preferência a região.
		
		\item
		Premium
		
		Modelo de assinatura mensal onde o usuário paga um valor e terá 5 dispositivos vinculado à assinatura. Ao assinar, a equipe instalará os softwares para acesso remoto e, quando o usuário não conseguir resolver por conta própria o problema, será atendido via chat ou WhatsApp. Para atendimento técnico, os técnicos são escolhidos de forma aleatória dando preferência a região e escolha de técnicos mais bem avaliados.
		
	\end{enumerate}

\section{Avaliação da visita técnica.}

	\begin{enumerate}
		
		\item
		Botão de Resolução (Sim ou Não)
		
		Após a conclusão da visita técnica aparecerá para o cliente uma pergunta se o problema foi resolvido com dois botões (Sim e Não), caso tenha resolvido apresentará a mensagem de dúvidas, sugestões ou reclamações. Em caso de não solução o incidente voltará a ser reportado para o time de atendimento técnico analisar o caso. 
		
		\item
		Avaliação da visita técnica
		
		Funcionalidade de avaliação da visita técnica do cliente com critérios de nota:(1 muito pouco satisfeito, 2 pouco satisfeito, 3 regular, 4 satisfeito, 5 muito satisfeito), o qual os técnicos mais bem avaliados terão mais chances de receber um contato de visita técnica. Existindo campos também para sugestões e reclamações.
		
	\end{enumerate}

\section{Modo de visualização}

	\begin{enumerate}
	
		\item
		Modo de aplicação em WEB
		
		O serviço será disponibilizado em formato de aplicação WEB acessível em qualquer navegador com interação facilitada.
		
		\item
		Avaliação da visita técnica.
		
		O serviço será disponibilizado em formato de aplicação WEB acessível em qualquer navegador com interação facilitada.
		
	\end{enumerate}

\section{Avaliação de visita técnica e clientes}

	\begin{enumerate}
		
		\item Avaliação da visita técnica.
		
		Funcionalidade de avaliação da visita técnica do cliente com critérios de nota:(1 muito pouco satisfeito, 2 pouco satisfeito, 3 regular, 4 satisfeito, 5 muito satisfeito), o qual os técnicos mais bem avaliados terão mais chances de receber um contato de visita técnica. Existindo campos também para sugestões e reclamações.
	
	
		\item Avaliação da visita técnica.
		
		Funcionalidade de avaliação da visita técnica do cliente com critérios de nota:(1 muito pouco satisfeito, 2 pouco satisfeito, 3 regular, 4 satisfeito, 5 muito satisfeito), o qual os técnicos mais bem avaliados terão mais chances de receber um contato de visita técnica. Existindo campos também para sugestões e reclamações.
	
		\item Avaliação do cliente.
		
		Funcionalidade de avaliação da visita técnica do cliente com critérios de nota:(1 muito pouco satisfeito, 2 pouco satisfeito, 3 regular, 4 satisfeito, 5 muito satisfeito), opção de classificar o nível de conhecimento técnico (1 nenhum conhecimento, 2 poucos conhecimento, 3 regular, 4 bons conhecimentos, 5 conhecimentos avançados).
	
	\end{enumerate}

\section{Relatórios}

	\begin{enumerate}

		\item 
		Relatório de Incidentes (Classificação, gerencial, traçar perfil de recorrência de incidente e comportamento).
		
		Geração de relatório de incidentes de nível gerencial, com demonstração de informações de classificação, recorrência de incidentes e comportamento dos usuários do FAQ.
	
		\item 
		Painel de indicadores
		
		Criação de painel de indicadores de consultas e comportamentos mais frequentes para melhoria da solução.

	\end{enumerate}
	
	