\chapter{DESENVOLVIMENTO}
	\section{Wireframes}
	Os wireframes criados foram divididos em seis telas: seção sobre, faq, planos, login, portal e novo chamado. 
	    \subsection{Seção sobre} 
    A primeira seção da nossa landing page como é mostrado na \autoref{lp-sobre-a-empresa}, trará as principais informações sobre nossa empresa, destacando em dois botões os principais objetivos e produtos da aplicação. 
    
    \begin{figure}[h]
        \caption{Seção sobre}
        \centering % para centralizarmos a figura
        \label{lp-sobre-a-empresa}
        \includegraphics[width=15cm]{LaTeX/metaversoIFSP/anexos/about-section.png} % leia abaixo
        \fonte{Os autores}
    \end{figure}
    
\subsection{FAQ}
    A segunda seção da nossa página principal, será o nosso produto FAQ(gratuito), conforme a \autoref{lp-faq}.
 
    \begin{figure}[!h]
        \caption{FAQ}
        \centering % para centralizarmos a figura
        \label{lp-faq}
        \includegraphics[width=15cm]{LaTeX/metaversoIFSP/anexos/faq.png} % leia abaixo
        \fonte{Os autores}
    \end{figure}
\newpage
\subsection{Planos} 
    A \autoref{lp-planos} apresenta a tela de pricing, na qual trazemos as opções disponíveis dos planos da nossa plataforma.
    \begin{figure}[!h]
        \caption{Planos}
        \centering % para centralizarmos a figura
        \label{lp-planos}
        \includegraphics[width=15cm]{LaTeX/metaversoIFSP/anexos/plans.png} % leia abaixo
        \fonte{Os autores}
    \end{figure}
\newpage
\subsection{Login} 
    A \autoref{lp-login} mostra a tela de login, onde o usuário poderá se autenticar para acessar seu portal interno da aplicação.
    \begin{figure}[h]
        \caption{Login}
        \centering % para centralizarmos a figura
        \label{lp-login}
        \includegraphics[width=15cm]{LaTeX/metaversoIFSP/anexos/login.png} % leia abaixo
        \fonte{Os autores}
    \end{figure}
    
    
\subsection{Portal (Dashboard)} 
    A \autoref{lp-dashboard} mostra o portal do usuário(Dashboard), focado na tela de chamados do cliente.
    \begin{figure}[h]
        \caption{Portal (Dashboard)}
        \centering % para centralizarmos a figura
        \label{lp-dashboard}
        \includegraphics[width=15cm]{LaTeX/metaversoIFSP/anexos/dashboard-chamados.png} % leia abaixo
        \fonte{Os autores}
    \end{figure}
    
\subsection{Novo Chamado} 
        A \autoref{lp-novo-chamado} apresenta o modal com formulário para abertura de novos chamados pelo usuário final.
    \begin{figure}[h]
        \caption{Novo Chamado}
        \centering % para centralizarmos a figura
        \label{lp-novo-chamado}
        \includegraphics[width=15cm]{LaTeX/metaversoIFSP/anexos/novo-chamado.png} % leia abaixo
        \fonte{Os autores}
    \end{figure}
%	\section{Avaliação SSL}
%	\section{Análise de resposta HTTP}
%	\section{Análise HTML}
%	\section{Métricas do Projeto}
%	\section{Escolhas e Descartes}
%	\subsection{Funcionalidades descartadas}
%	\subsection{Mudanças na infraestrutura}
%	\section{Problemas enfrentados}
	
%	\section{Estatísticas SVN}
	
	\section{Links do Projeto}
	\newcommand{\urlGithub}{https://github.com/MetaversoIFSP/}
	\newcommand{\urlYoutube}{https://www.youtube.com/channel/UCNXlPi5ADeGx1tZbMo_xOqw/}
	\newcommand{\urlSVN}{https://svn.spo.ifsp.edu.br/viewvc/A6PGP/S202201-PI-NOT/Metaverso/}
	\newcommand{\urlBlog}{https://metaversoifsp.blogspot.com/}
	Nesta seção estão disponíveis os \textit{links} de acesso pertinentes ao projeto.
	
    \begin{figure}[h]
        \centering
        \captionsetup{justification=centering}
        \qrcode{\urlGithub}
        \caption{\textit{QR Code} - Repositório remoto GitHub \\ \url{\urlGithub}}
        \fonte{Os autores}
    \end{figure}
    
    \begin{figure}[h]
        \centering
        \captionsetup{justification=centering}
        \qrcode{\urlYoutube}
        \caption{\textit{QR Code} - Canal Metaverso no Youtube \\ \url{\urlYoutube}}
        \fonte{Os autores}
    \end{figure}
    
    \begin{figure}[h]
        \centering
        \captionsetup{justification=centering}
        \qrcode{\urlSVN}
        \caption{\textit{QR Code} - Repositório remoto Subversion\\
        \url{\urlSVN}}
        \fonte{Os autores}
    \end{figure}
    
    \begin{figure}[h]
        \centering
        \captionsetup{justification=centering}
        \qrcode{\urlBlog}
        \caption{\textit{QR Code} - Blog Metaverso \\ \url{\urlBlog}}
        \fonte{Os autores}
    \end{figure}  



