Os requisitos funcionais são descritos na \hyperlink{tabela-requisitos-funcionais}{Tabela 6}.

\begin{table}[h]
\hypertarget{tabela-requisitos-funcionais}
	\centering\footnotesize
	\caption{Requisitos Funcionais}
	\resizebox {1.0 \textwidth }{!}{
        \begin{tabular}{|l|p{8cm}|p{8cm}|}
        \hline
        \textbf{Identificador}            & \textbf{Requisito} & \textbf{Descrição}                                                            \\ \hline
        RF01 & O sistema deve permitir o cadastro de técnicos  & O sistema deve permitir que técnicos consigam realizar o cadastro para adicionar os dados de identificação                        \\ \hline
        RF02 & O sistema deve permitir cadastro de clientes facilitada  & O sistema deve permitir que clientes consigam realizar o cadastro com o mínimo de dados necessários e, posteriormente, incluir dados adicionais                       \\ \hline
        RF03 & O sistema deve permitir abertura de chamado  & O cliente deve conseguir detalhar o problema que não conseguiram resolver pelo FAQ             \\ \hline
        RF04 & O sistema deve permitir a visualização de histórico do chamado  & O usuário poderá acompanhar histórico de seus chamados, andamento e avaliações           \\ \hline
        RF05 & O sistema deve permitir cadastro de instruções no FAQ  & O sistema deve permitir que novas soluções sejam criadas, editadas, suprimidas ou deletadas no FAQ \\ \hline
        RF06 & O sistema deve permitir a realização do login  & O sistema deve permitir a realização do login para os usuários que estão cadastrados na plataforma \\ \hline
        RF07 & O sistema deve permitir visualização de histórico de incidentes  & O sistema deve permitir que técnicos tenham a visualização ao histórico de problemas e tratativas que foram aplicadas para um cliente ao realizar um atendimento 		\\ \hline
        \end{tabular}
    }    
\end{table}
\fonte{Os autores}