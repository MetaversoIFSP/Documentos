% ---
% Capitulo de revisão de literatura
% ---
%\chapter{Revisão da Literatura}
Nesta seção, serão abordados o princípio do ITSM para melhor compreensão do projeto e solução proposta.
% \explicacao{Todos trabalhos devem possuir a revisão de literatura onde são abordados os estudos feitos com base da literatura (livros, artigos acadêmicos, publicações em periódicos), todos elementos devem ser referenciados por citações.}

% \explicacao{Diversas referencias utilizadas nesse modelo não deveriam ser utilizadas diretamente em um trabalho acadêmico, mas estão aqui para demonstrar de forma mais clara alguns pontos importantes sobre desenvolvimento de projetos}

% \explicacao{Cada parágrafo da revisão de literatura deve apresentar uma ideia com base em uma referencia }

% \explicacao{Copiar e colocar é plágio. Exceto em casos muito específicos onde utilizamos a citação direta você deve escrever com suas palavras (seu entendimento, parafrasear) o que os autores escreveram na publicação original }

% \explicacao{Não são abordados aqui itens técnicos que normalmente são vistos em disciplinas anteriores do curso (UML, banco de dados, metodologias de gerenciamento de projeto etc...), esses elementos podem receber citações nos outros capítulos do trabalho. Essa regra não se aplica aos trabalhos de pós graduação quando o tema estiver relacionado a conceitos técnicos.}


%\preencheComTexto
% ---

% ---
%\section{ITSM}

%\preencheComTexto

%\preencheComTexto

%\section{Assunto 3}
%\preencheComTexto

%\section{Assunto X}
%\preencheComTexto
% ---
%\section{Metodologia}