\chapter{Gerenciamento do Projeto}

	\section{Metodologia utilizada}
		
		A metodologia aplicada no desenvolvimento da ferramenta fez usos de práticas combinadas das metodologias Scrum e Kamban, com alterações pontuais para melhor adequação aos prazos da disciplina e práticas familiarizadas pelos componentes do grupo.

		Assim, a equipe aaderiu aos seguintes artefatos do Scrum:

		\begin{itemize}
			\item 
				Sprint Planning

				Em conjunto com as práticas do scrum, utilizamos o trello como ferramenta kanban para organizar as tarefas sendo desenvolvidas.

			\item 
				Diagrama de caso de usos
				
				A fim de minimizar a questão de limitação de tempo, aplicou-se o diagrama de caso de uso. Assim, contorna-se a necessidade de refinar novas tarefas durante o desenvolvimento.

		\end{itemize}
		
	\section{Equipe e Projeto}

		A equipe é composta por seis integrantes, a fim de cumprir o requisito de conclusão do curso de Análise e Desenvolvimento de Sistemas.

		As tarefas foram distribuídas entre os integrantes, mas não restrita ao indivíduo. Essa distribuição tem por objetivo priorizar e nomear pontos focais.
		
		\begin{itemize}
			\item 
				Henrique Hiromi Shimada
				
				Atuou como desenvolvedor de software por aproximadamente um ano, utilizando principalmente as linguagens Java e Python, serviços gerenciados da nuvem AWS. Atualmente, atua como consultor de entrega de pacotes em soluções corporativas na área dados IBM.

			\item 
				Isabela Souza Duarte



			\item 
				Mateus Souza da Silva



			\item 
				Vinicius Gomes Moreira



			\item 
				Welen Mota Sousa


				
		\end{itemize}

	\subsection{Comunicação do Projeto}

		O progresso das etapas do projeto foi publicado em um blog, de acordo com a orientação dos professores.

		\begin{table}[h] % manter com o h para que a tabela não seja ancorado no topo
			\centering\footnotesize
			\caption{\label{canais} Canais de comunicação}
			\begin{tabular}{|l|l|}
			\hline
				SVN     & https://svn.spo.ifsp.edu.br/viewvc/A6PGP/S202201-PI-NOT/Metaverso/ \\ \hline
				Blog    & https://metaversoifsp.blogspot.com/                                \\ \hline
				Youtube & https://www.youtube.com/channel/UCNXlPi5ADeGx1tZbMo\_xOqw          \\ \hline
			\end{tabular}
		\end{table}


	\subsection{Divisão de Tarefas}

		\begin{table}[h] % manter com o h para que a tabela não seja ancorado no topo
			\centering\footnotesize
			\caption{\label{analise} Organização da Equipe}
			%\resize
			\resizebox {.75 \textwidth }{!}{
				\begin{tabular}{|l|l|l|l|l|l|}
					\hline
					Tarefa         			 & Henrique & Isabela & Mateus & Vinicius & Welen \\ \hline
					Front-End                &		&	X	&	X	&	X	&		\\ \hline
					Back-End                 &	X	&		&		&		&		\\ \hline
					Banco de Dados           &	X	&		&		&	X	&		\\ \hline
					Documentação e Modelagem &	X	&	X	&		&		&		\\ \hline
					Vídeos                   &	X	&	X	&	X	&	X	&	X	\\ \hline
					Blog                     &	X	&	X	&	X	&	X	&	X	\\ \hline
				\end{tabular}
			}
		\end{table}
		\fonte{Estudo feito pelos autores}

	\section{Sprints}

		Seguindo a recomendação dos orientadores, as sprints foram ajustadas para ocorrerem em uma semana.

		\section{Gestão do Tempo}
    \begin{table}[]
        \begin{tabular}{|l|l|p{8cm}|}
        \hline
        Sprint                & Duração & Atividades                                                            \\ \hline
        Sprint 1 (16/03/2022) & 7 dias  & Brainstorm de ideias e escolha da proposta                            \\ \hline
        Sprint 2 (23/03/2022) & 7 dias  & Descrição das Funcionalidades do Projeto ITSM                         \\ \hline
        Sprint 3 (30/03/2022) & 7 dias  & Elaboração da arquitetura de software ITSM                            \\ \hline
        Sprint 4 (06/04/2022) & 7 dias  & Definição e Elaboração de artefatos de Negócios e Software            \\ \hline
        Sprint 5 (13/04/2022) & 7 dias  & Correção de Artefatos e Elaborado Histórias de Usuário e Casos de uso \\ \hline
        Sprint 6 (20/04/2022) & 7 dias  & Melhorias nos Diagramas, Caso de USo, Requisito FUncionais e Não Funcionais \\ \hline
        Sprint 7 (27/04/2022) & 7 dias  & Entrega do Desenho da Aplicação e Elaboração da Apresentação 		\\ \hline
        Sprint 8 (04/05/2022) & 7 dias  & Correção dos Diagramas, Caso de Uso, Requisitos e Apresentação do Desenho da Aplicação \\ \hline
        Sprint 9 (11/05/2022) & 7 dias  & Apresentação POC e correções da Documentação 				\\ \hline
        Sprint 10 (18/05/2022) & 7 dias  & A DEFINIR 								\\ \hline
        Sprint 11 (25/05/2022) & 7 dias  & A DEFINIR 								\\ \hline
        Sprint 12 (01/06/2022) & 7 dias  & A DEFINIR 								\\ \hline
        Sprint 13 (08/06/2022) & 7 dias  & A DEFINIR 								\\ \hline
        Sprint 14 (15/06/2022) & 7 dias  & A DEFINIR 								\\ \hline
        Sprint 15 (22/06/2022) & 7 dias  & A DEFINIR 								\\ \hline
        \end{tabular}
    \end{table}

		%\subsection{Sprint 1}