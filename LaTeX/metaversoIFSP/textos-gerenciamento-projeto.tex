\chapter{Gerenciamento do Projeto}

	\section{Metodologia utilizada}
		
		A metodologia aplicada no desenvolvimento da ferramenta fez usos de práticas combinadas das metodologias \textit{Scrum} e \textit{Kanban}, com alterações pontuais para melhor adequação aos prazos da disciplina e práticas familiarizadas pelos componentes do grupo.

		Assim, a equipe aderiu aos seguintes artefatos do \textit{Scrum}:

		\begin{itemize}
			\item 
				\textit{Sprint Planning}

				Em conjunto com as práticas do \textit{Scrum}, utilizamos o \textit{Trello} como ferramenta \textit{Kanban}para organizar as tarefas sendo desenvolvidas.

			\item 
				Diagrama de caso de usos
				
				A fim de minimizar a questão de limitação de tempo, aplicou-se o diagrama de caso de uso. Assim, contorna-se a necessidade de refinar novas tarefas durante o desenvolvimento.

		\end{itemize}
		
	\section{Equipe e Projeto}

		A equipe é composta por cinco integrantes, a fim de cumprir o requisito de conclusão do curso de Análise e Desenvolvimento de Sistemas.

		As tarefas foram distribuídas entre os integrantes, mas não restrita ao indivíduo. Essa distribuição tem por objetivo priorizar e nomear pontos focais.
		
		\begin{itemize}
			\item 
				Henrique Hiromi Shimada
				
				É o ponto focal da equipe para o desenvolvimento do \textit{back-end} da aplicação. Com experiência em Java, tem propriedade no assunto e aborda as melhores práticas para desenvolvimento do mesmo. Além disso, contribuiu com a documentação e foi o principal mantenedor do repositório no SVN.
                
			\item 
				Isabela Souza Duarte
				
                Principal responsável pelas manutenções da documentação, guiou a equipe durante a dissertação e alertando dos pontos de atenção. Usou também de sua expertise em programação \textit{front-end} para auxiliar a equipe durante o desenvolvimento.  Além disso, contribuiu nas pesquisas que nos serviram de base para o desenvolvimento do projeto.

			\item 
				Mateus Souza da Silva

                Principal desenvolvedor \textit{front-end}, usou sua experiência na área para criar os \textit{wireframes} da aplicação, que serviriam de guia para o desenvolvimento do mesmo, que também seria conduzido por ele. Também contribuiu com a documentação.

			\item 
				Vinicius Gomes Moreira
                
                \textit{Product Owner} da equipe, é responsável por agregar valor à solução e detalhar as necessidades que a mesma deverá atender. Construiu a estrutura de diagramas da solução, visando atender as necessidades estipuladas. Além disso, atuou na documentação, desenvolvimento da prova de conceito e preparação de ambientes para o desenvolvimento do MVP. 
                


			\item 
				Welen Mota Sousa

                \textit{Scrum Master} da equipe, garantiu que as cerimônias fossem cumpridas de forma produtiva, além de ser a principal responsável pelas pesquisas que guiariam o projeto. Assim como os demais membros, contribuiu com a documentação.
				
		\end{itemize}

	\subsection{Comunicação do Projeto}

		O progresso das etapas do projeto foi publicado em um \textit{blog}, de acordo com a orientação dos professores orientadores, conforme se encontra na \hyperlink{canais-comunicacao}{Tabela 1}.

		\begin{table}[h] % manter com o h para que a tabela não seja ancorado no topo
			\hypertarget{canais-comunicacao}
			\centering\footnotesize
			\caption{\label{canais} Canais de comunicação}
			\begin{tabular}{|l|c|}
			\hline
			    \textbf{Canais} & \textbf{Links} \\ \hline
				SVN     & https://svn.spo.ifsp.edu.br/viewvc/A6PGP/S202201-PI-NOT/Metaverso/ \\ \hline
				\textit{Blog}   & https://metaversoifsp.blogspot.com/                                \\ \hline
				\textit{Youtube} & https://www.youtube.com/channel/UCNXlPi5ADeGx1tZbMo\_xOqw          \\ \hline
			\end{tabular}
		\end{table}
        \fonte{Os autores}

	\subsection{Divisão de Tarefas}
	Todos da equipe trabalharão no desenvolvimento do projeto conforme a \hyperlink{organizacao-equipe}{Tabela 2}.


		\begin{table}[h] % manter com o h para que a tabela não seja ancorado no topo
		\hypertarget{organizacao-equipe}
			\centering\footnotesize
			\caption{\label{analise} Organização da Equipe}
			%\resize
			\resizebox {.75 \textwidth }{!}{
				\begin{tabular}{|l|c|c|c|c|c|}
					\hline
					\textbf{Tarefa}         			 & 	\textbf{Henrique} & 	\textbf{Isabela} & 	\textbf{Mateus} & 	\textbf{Vinicius} & \textbf{Welen} \\ \hline
		            \textit{Front-End}                       &		&	X	&	X	&	X	&		\\ \hline
		        	\textit{Back-End}	                &	X	&		&		&		&		\\ \hline
		            Banco de Dados           &	X	&		&		&	X	&		\\ \hline
					Documentação e Modelagem &	X	&	X	&		&		&		\\ \hline
					Vídeos                   &	X	&	X	&	X	&	X	&	X	\\ \hline
					\textit{Blog}                    &	X	&	X	&	X	&	X	&	X	\\ \hline
				\end{tabular}
			}
		\end{table}
		\fonte{Os autores}

	\section{\textit{Sprints}}

		Seguindo a recomendação dos orientadores, as sprints foram ajustadas para ocorrerem em uma semana.

		\section{Gestão do Tempo}
    \begin{table}[h] % manter com o h para que a tabela não seja ancorado no topo
        \centering\footnotesize
        \caption{\label{sprints} Organização das sprints}
        \begin{tabular}{|l|l|p{9cm}|}
            \hline
            Sprint & Duração & Atividades \\ \hline
            Sprint 1 (16/03/2022) & 7 dias & Brainstorm de ideias e escolha da proposta \\ \hline
            Sprint 2 (23/03/2022) & 7 dias & Descrição das Funcionalidades do Projeto ITSM \\ \hline
            Sprint 3 (30/03/2022) & 7 dias & Elaboração da arquitetura de software ITSM \\ \hline
            Sprint 4 (06/04/2022) & 7 dias & Definição e Elaboração de artefatos de Negócios e Software \\ \hline
            Sprint 5 (13/04/2022) & 7 dias & Correção de Artefatos e Elaborado Histórias de Usuário e Casos de uso\\ \hline
        \end{tabular}
		\fonte{Estudo feito pelos autores}
    \end{table}

		%\subsection{Sprint 1}