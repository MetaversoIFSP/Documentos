\chapter{Gerenciamento do Projeto}

	\section{Metodologia utilizada}
		
		A metodologia aplicada no desenvolvimento da ferramenta fez usos de práticas combinadas das metodologias Scrum e Kamban, com alterações pontuais para melhor adequação aos prazos da disciplina e práticas familiarizadas pelos componentes do grupo.

		Assim, a equipe aaderiu aos seguintes artefatos do Scrum:

		\begin{itemize}
			\item 
				Sprint Planning

				Em conjunto com as práticas do scrum, utilizamos o trello como ferramenta kanban para organizar as tarefas sendo desenvolvidas.

			\item 
				Diagrama de caso de usos
				
				A fim de minimizar a questão de limitação de tempo, aplicou-se o diagrama de caso de uso. Assim, contorna-se a necessidade de refinar novas tarefas durante o desenvolvimento.

		\end{itemize}
		
	\section{Equipe e Projeto}

		A equipe é composta por seis integrantes, a fim de cumprir o requisito de conclusão do curso de Análise e Desenvolvimento de Sistemas.

		As tarefas foram distribuídas entre os integrantes, mas não restrita ao indivíduo. Essa distribuição tem por objetivo priorizar e nomear pontos focais.
		
		\begin{itemize}
			\item 
				Henrique Hiromi Shimada
				
				Atuou como desenvolvedor de software por aproximadamente um ano, utilizando principalmente as linguagens Java e Python, serviços gerenciados da nuvem AWS. Atualmente, atua como consultor de entrega de pacotes em soluções corporativas na área dados IBM.

			\item 
				Isabela Souza Duarte



			\item 
				Mateus Souza da Silva



			\item 
				Vinicius Gomes Moreira



			\item 
				Welen Mota Sousa


				
		\end{itemize}

	\subsection{Comunicação do Projeto}

		O progresso das etapas do projeto foi publicado em um blog, de acordo com a orientação dos professores.

	\subsection{Divisão de Tarefas}

		\begin{table}[]
			\begin{tabular}{|l|l|}
			\hline
				SVN     & https://svn.spo.ifsp.edu.br/viewvc/A6PGP/S202201-PI-NOT/Metaverso/ \\ \hline
				Blog    & https://metaversoifsp.blogspot.com/                                \\ \hline
				Youtube & https://www.youtube.com/channel/UCNXlPi5ADeGx1tZbMo\_xOqw          \\ \hline
			\end{tabular}
		\end{table}

		\begin{table}[]
			\begin{tabular}{|l|l|l|l|l|l|}
				\hline
				Tarefa         & Henrique & Isabela & Mateus & Vinicius & Welen \\ \hline
				Front-end      &          &         &        &          &       \\ \hline
				Back-end       &          &         &        &          &       \\ \hline
				Banco de Dados &          &         &        &          &       \\ \hline
				Infraestrutura &          &         &        &          &       \\ \hline
				Testes         &          &         &        &          &       \\ \hline
				Cobertura      &          &         &        &          &       \\ \hline
				Documentação   &          &         &        &          &       \\ \hline
				Blog           &          &         &        &          &       \\ \hline
				Vídeos         &          &         &        &          &       \\ \hline
				Gerenciamento  &          &         &        &          &       \\ \hline
			\end{tabular}
			\end{table}

	\section{Sprints}

		Seguindo a recomendação dos orientadores, as sprints foram ajustadas para ocorrerem em uma semana.

	\subsection{Sprint 1}