\chapter[Tecnologias]{Tecnologias}
As tecnologias adotadas foram baseadas de acordo com a experiência e afinidade dos membros da equipe. Desse forma, as tecnologias escolhidas para cada área são:
	\section{Modelo de trabalho}
	
	Considerando o prazo de entrega e metodologias de trabalho estudadas durante o curso, bem como de comum conhecimento profissional dos elementos do grupo, decidiu-se o emprego dos framework Kanban e Scrum para o desenvolvimento do projeto, alinhado {\`a}s práticas do DevOps.

	\section{Infra estrutura}
	
	A aplica{\c{c}}ão será mantida em ambiente cloud na modalidade de IAAS (Infrastructure as a Service, ou Infraestrutura como servi{\c{c}}o), utilizando os servi{\c{c}}os do provedor Microsoft Azure.
	
	\section{Back-end}
	
	Considerando a maturidade das bibliotecas da linguagem java, o amplo suporte que existe da comunidade e a experiência de desenvolvimento dos integrantes do grupo, foi escolhida a linguagem Java para sustentar a aplica{\c{c}}ão de back-end desenvolvido em Java e banco de dados MySql para persistência.
	
	\section{Front-end}
	
	Tendo em vista, a questão desempenho e curva de aprendizagem, o Front-End da aplicação será desenvolvido utilizando a biblioteca JavaScript React, utilizando o superset Typescript - focando no recurso de tipagem que ele nos traz, o que auxilia tanto no desenvolvimento quanto na manutenção e refatoração posterior. Será utilizado também bibliotecas focadas em estilo, tais como Material Ui, a fim de se manter um padrão nos nossos componentes visuais. 