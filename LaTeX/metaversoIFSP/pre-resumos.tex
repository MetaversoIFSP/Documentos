% ---
% RESUMOS
% ---

% resumo em português
\setlength{\absparsep}{18pt} % ajusta o espaçamento dos parágrafos do resumo
\begin{resumo}
\todo[inline]{fazer o seu resumo, ele só é feito depois que o documento está terminado
\newline
\newline os itens em negrito estão aqui para ressaltar detalhes que devem ser seguidos, mas não se utiliza o negrito em um resumo}

Visando atender um nicho de mercado, desenvolvemos a solução de ITSM que deve satisfazer um problema recorrente em pequenas empresas e para usuários menos instruídos em relação à computadores e tecnologias. Estes usuários geralmente sabem fazer apenas aquilo que executam todos os dias, como: ligar o computador, fazer uma impressão, pesquisar vídeos, etc. Ao terem algum problema em seu dispositivo, geralmente não sabem como proceder. Assim, a ferramenta desenvolvida poderia auxiliar os usuários através de um FAQ intuitivo, onde através de opção de fácil compreensão, poderiam chegar até a solução de seu problema. Além disso, existem as opcões de atendimento técnico, via chat ou agendamento de visita técnica. Assim, os mesmos tem um canal de solução rápida para os problemas técnicos enfrentados no dia a dia. Para a solução, dispomos dos modelos de atendimento previamente mencionados, além de categorizar os problemas e soluções conforme relevância de resolução, facilitando ainda mais ao usuário em localizar seu problema e respectiva solução. 

 %De acordo com a norma \citetitle{NBR6028:2003} (3.1-3.2) \index{NBR6028}, o resumo\index{resumo} deve ressaltar o contexto, o objetivo, o método, os resultados e as conclusões do documento (portanto deve ser escrito por ultimo). A ordem e a extensão destes itens dependem do tipo de resumo (informativo ou indicativo) e do  tratamento que cada item recebe no documento original. O resumo \textbf{deve ter um paragrafo único} e deve \textbf{ter entre 150 (cento e cinquenta) e 500 palavras para trabalhos acadêmicos ou entre 100 e 250 para artigos de periódicos}. O resumo deve ser  precedido da referência do documento, com exceção do resumo inserido no  próprio documento. (\ldots) As palavras-chave devem figurar logo abaixo do resumo, antecedidas da expressão \textbf{Palavras-chave}:, separadas entre si por ponto e finalizadas também por ponto.

 \textbf{Palavras-chaves}: problemas técnicos. agendamento. chamados. FAQ.
\end{resumo}

% resumo em inglês
\begin{resumo}[Abstract]
 \begin{otherlanguage*}{english}

   We designed the ITSM solution to serve a niche market and to address a reoccurring problem in small businesses and for less educated users in regard to computers and technologies. These users typically only know how to accomplish basic tasks like turning on the computer, printing, and browsing for movies. When people experience an issue with their equipment, they are often at a loss for what to do. As a result, the produced application could assist customers through an informative FAQ, where they might find a solution to their problem through an easy-to-understand alternative. Additionally, technical support is available through chat or by scheduling a technical visit. As a result, they have a quick response route for technical issues that arise on a daily basis.

\todo[inline]{fazer tradução do resumo, não utilizar tradução automática}

\todo[inline]{Cuidado com termos que só fazem sentido na língua portuguesa, o texto deve ser ajustado para fazer sentido aos leitores que não conhecem a língua portuguesa}

   \vspace{\onelineskip}

   \noindent 
   \textbf{Keywords}: Technical problems. Scheduling. Technical calls. FAQ.
 \end{otherlanguage*}
\end{resumo}