%% Adaptado a partir de :
%%    abtex2-modelo-trabalho-academico.tex, v-1.9.2 laurocesar
%% para ser um modelo para os trabalhos no IFSP-SPO

\documentclass[
    % -- opções da classe memoir --
    12pt,               % tamanho da fonte
    openright,          % capítulos começam em pág ímpar (insere página vazia caso preciso)
    %twoside,            % para impressão em verso e anverso. Oposto a oneside
    oneside,
    a4paper,            % tamanho do papel. 
    % -- opções da classe abntex2 --
    %chapter=TITLE,     % títulos de capítulos convertidos em letras maiúsculas
    %section=TITLE,     % títulos de seções convertidos em letras maiúsculas
    %subsection=TITLE,  % títulos de subseções convertidos em letras maiúsculas
    %subsubsection=TITLE,% títulos de subsubseções convertidos em letras maiúsculas
    % Opções que não devem ser utilizadas na versão final do documento
    %draft,              % para compilar mais rápido, remover na versão final
    %MODELO,             % indica que é um documento modelo então precisa dos geradores de texto
    %TODO,               % indica que deve apresentar lista de pendencias 
    % -- opções do pacote babel --
    english,            % idioma adicional para hifenização
    brazil              % o último idioma é o principal do documento
   ]{ifsp-spo-inf-ctds}
\graphicspath{ {./images/} }
% ---

% --- 
% CONFIGURAÇÕES DE PACOTES
% --- 
%\usepackage{etoolbox}
%\patchcmd{\thebibliography}{\chapter*}{\section*}{}{}


% ---
% Informações de dados para CAPA e FOLHA DE ROSTO
% ---
\titulo{Metaverso - Ferramenta de Chamados de TI (ITSM) para pessoas físicas e pequenas empresas}

% Trabalho individual
%\autor{JOSÉ BRAZ DE ARAUJO}

% Trabalho em Equipe
% ver também https://github.com/abntex/abntex2/wiki/FAQ#como-adicionar-mais-de-um-autor-ao-meu-projeto
    \renewcommand{\imprimirautor}{
        \begin{tabular}{lr}
        %CEZAR GODOY NASCIMENTO	& SP3040755 \\
        HENRIQUE HIROMI SHIMADA & SP3039421 \\
        ISABELA SOUZA DUARTE	& SP3030083 \\
        MATEUS SOUZA DA SILVA	& SP3022374 \\
        VINICIUS GOMES MOREIRA	& SP3039587 \\
        WELEN MOTA SOUSA	& SP146616X \\
        \end{tabular}
}


\tipotrabalho{Projeto da Disciplina Projeto Integrado I}

\disciplina{PI1A5 - Projeto Integrado I}

\preambulo{Proposta de projeto para disciplina PI1A5}

\data{MARÇO DE 2022}

% Definir o que for necessários e comentar o que não for necessário
% Utilizar o Nome Completo, abntex tem orientador e coorientador
% então vão ser utilizados na definição de professor
\renewcommand{\orientadorname}{Professor:}
\orientador{JOSE BRAZ DE ARAUJO}
\renewcommand{\coorientadorname}{Professor:}
\coorientador{MARCELO TAVARES DE SANTANA}



% ---


% ---
% Configurações de aparência do PDF final


% informações do PDF
\makeatletter
\hypersetup{
        %pagebackref=true,
        pdftitle={\@title}, 
        pdfauthor={\@author},
        pdfsubject={\imprimirpreambulo},
        pdfcreator={LaTeX with abnTeX2},
        pdfkeywords={abnt}{latex}{abntex}{abntex2}{trabalho acadêmico}, 
        colorlinks=true,            % false: boxed links; true: colored links
        linkcolor=blue,             % color of internal links
        citecolor=blue,             % color of links to bibliography
        filecolor=magenta,              % color of file links
        urlcolor=blue,
        bookmarksdepth=4
}
\makeatother
% --- 

% ---

% ----
% Início do documento
% ----
\begin{document}

% Retira espaço extra obsoleto entre as frases.
\frenchspacing 

\pretextual

% ---
% Capa - Para proposta a folha de rosto é suficiente pois é mais completa.
% ---
\imprimirfolhaderosto
% ---

% ----------------------------------------------------------
% ELEMENTOS TEXTUAIS
% ----------------------------------------------------------
\textual

\chapter[Introdução]{Introdução}

	De acordo com a PNAD de 2019 (\citeauthor{PNAD:2019}), 82,7\% dos domicílios brasileiros contam com acesso à internet. Mesmo tendo acesso de banda larga em 80,2\% dos dispositivos móveis do país, apenas 45,1\% os domicílios da amostra têm acesso via computador. Assim, entendemos que embora muitos dos usuários têm alguma fluência em aplicativos móveis, as diferenças de interfaces pode ser um desafio para o usuário que têm tarefas diferentes das desempenhadas em aplicativos móveis.

	Ainda, considerando que existem diversas soluções corporativas de suporte, em contraste, para esses usuários, as soluções que mais se apresentam são fóruns, que ainda que sejam gratuitas, ainda demandam algum trabalho e interação que os usuários alvo são pouco familiarizados ou têm compreensão limitados em relação à dinâmica de tais ferramentas.

	Assim, propomos uma ferramenta com capacidade de suportar os usuários que sentem necessidade de suporte personalizado.

\chapter[Objetivos]{Objetivos}

	O projeto elaborado propõe uma solução que atenda pessoas com baixa fluência em sistemas de computação em situações cotidianas em que seus dispositivos não funcionem de acordo com o esperado pelo usuário.
	
	Entende-se que a solução tem como alvo pessoas físicas e pequenas empresas, que normalmente têm acesso limitado ou nenhum a ferramentas tradicionais de suporte de tecnologia da informação (\textit{ITSM - Information Technology Service Management}).
	
	Para tanto, foram definidos os seguintes objetivos para criação do serviço: elaboração do mínimo produto viável e suas ferramentas essenciais, com pontos de checagem (\textit{check point}) para verificação do avanço da solução.

\section{Objetivo Principal}

	Prover assistência a problemas em sistemas computacionais domésticos ou de pequenas empresas para usuários finais com pouca ou nenhuma familiaridade a problemas cotidianos.

\section{Objetivos Secundários}

	Para que o produto de ITSM - \textit{Information Technology Service Management} seja considerado viável, será necessário que as seguintes ferramentas sejam disponibilizadas as funcionalidades a seguir: 
	
	\begin{itemize}
		
		\item
		Familiarizar o usuário com pouca habilidade para resolver problemas de soluções simples com computadores pra que possam ser independentes;
		
		\item
		Facilitar a identificação e aplicação de resolução a problemas com computadores;
		
		\item
		Orientar os gestores da ferramenta sobre as questões e dificuldades mais comuns;
		
		\item 
		Facilitar o processo de resolução em caso de atendimento por terceiro credenciado
		 
	\end{itemize}

\end{document}