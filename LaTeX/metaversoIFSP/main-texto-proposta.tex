%% Adaptado a partir de :
%%    abtex2-modelo-trabalho-academico.tex, v-1.9.2 laurocesar
%% para ser um modelo para os trabalhos no IFSP-SPO

\documentclass[
    % -- opções da classe memoir --
    12pt,               % tamanho da fonte
    openright,          % capítulos começam em pág ímpar (insere página vazia caso preciso)
    %twoside,            % para impressão em verso e anverso. Oposto a oneside
    oneside,
    a4paper,            % tamanho do papel. 
    % -- opções da classe abntex2 --
    %chapter=TITLE,     % títulos de capítulos convertidos em letras maiúsculas
    %section=TITLE,     % títulos de seções convertidos em letras maiúsculas
    %subsection=TITLE,  % títulos de subseções convertidos em letras maiúsculas
    %subsubsection=TITLE,% títulos de subsubseções convertidos em letras maiúsculas
    % Opções que não devem ser utilizadas na versão final do documento
    %draft,              % para compilar mais rápido, remover na versão final
    MODELO,             % indica que é um documento modelo então precisa dos geradores de texto
    %TODO,               % indica que deve apresentar lista de pendencias 
    % -- opções do pacote babel --
    english,            % idioma adicional para hifenização
    brazil              % o último idioma é o principal do documento
   ]{ifsp-spo-inf-ctds}
     
% ---

% --- 
% CONFIGURAÇÕES DE PACOTES
% --- 
%\usepackage{etoolbox}
%\patchcmd{\thebibliography}{\chapter*}{\section*}{}{}


% ---
% Informações de dados para CAPA e FOLHA DE ROSTO
% ---
\titulo{Metaverso - Ferramenta de Chamados de TI (ITSM) para pessoas físicas e pequenas empresas}

% Trabalho individual
%\autor{JOSÉ BRAZ DE ARAUJO}

% Trabalho em Equipe
% ver também https://github.com/abntex/abntex2/wiki/FAQ#como-adicionar-mais-de-um-autor-ao-meu-projeto
\renewcommand{\imprimirautor}{
\begin{tabular}{lr}
CEZAR GODOY NASCIMENTO	& SP3040755 \\
HENRIQUE HIROMI SHIMADA & SP3039421 \\
ISABELA SOUZA DUARTE	& SP3030083 \\
MATEUS SOUZA DA SILVA	& SP3022374 \\
VINICIUS GOMES MOREIRA	& SP3039587 \\
WELEN MOTA SOUSA		& SP146616X \\
\end{tabular}
}


\tipotrabalho{Projeto da Disciplina Projeto Integrado I}

\disciplina{PI1A5 - Projeto Integrado I}

\preambulo{Proposta de projeto para disciplina PI1A5}

\data{MARÇO DE 2022}

% Definir o que for necessário e comentar o que não for necessário
% Utilizar o Nome Completo, abntex tem orientador e coorientador
% então vão ser utilizados na definição de professor
\renewcommand{\orientadorname}{Professor:}
\orientador{JOSE BRAZ DE ARAUJO}
\renewcommand{\coorientadorname}{Professor:}
\coorientador{MARCELO TAVARES DE SANTANA}



% ---


% ---
% Configurações de aparência do PDF final


% informações do PDF
\makeatletter
\hypersetup{
        %pagebackref=true,
        pdftitle={\@title}, 
        pdfauthor={\@author},
        pdfsubject={\imprimirpreambulo},
        pdfcreator={LaTeX with abnTeX2},
        pdfkeywords={abnt}{latex}{abntex}{abntex2}{trabalho acadêmico}, 
        colorlinks=true,            % false: boxed links; true: colored links
        linkcolor=blue,             % color of internal links
        citecolor=blue,             % color of links to bibliography
        filecolor=magenta,              % color of file links
        urlcolor=blue,
        bookmarksdepth=4
}
\makeatother
% --- 

% ---

% ----
% Início do documento
% ----
\begin{document}

% Retira espaço extra obsoleto entre as frases.
\frenchspacing 

\pretextual

% ---
% Capa - Para proposta a folha de rosto é suficiente pois é mais completa.
% ---
\imprimirfolhaderosto
% ---

% ----------------------------------------------------------
% ELEMENTOS TEXTUAIS
% ----------------------------------------------------------
\textual

\chapter[Introdução]{Introdução}

De acordo com a PNAD de 2019 (\citeauthor{PNAD:2019}), 82,7\% dos domicílios brasileiros contam com acesso à internet.

Mesmo tendo acesso de banda larga em 80,2\% dos dispositivos móveis do país, apenas 45,1\% os domicílios da amostra têm acesso via computador. Assim, entendemos que embora muitos dos usuários têm alguma fluência em aplicativos móveis, as diferenças de interfaces pode ser um desafio para o usuário que têm tarefas diferentes das desempenhadas em aplicativos móveis.

Ainda, considerando que existem diversas soluções corporativas de suporte, em contraste, para esses usuários, as soluções que mais se apresentam são fóruns, que ainda que sejam gratuitas, ainda demandam algum trabalho e interação que os usuários alvo são pouco familiarizados ou têm compreensão limitados em relação à dinâmica de tais ferramentas.

Assim, propomos uma ferramenta com capacidades de suportar os usuários que sentem necessidade de suporte personalizado.

\section{Problema}
Problema: dificuldade de resolver problemas corriqueiros incomodam ou impedem o uso esperado do dispositivo

\subsection{Personas}
Usuários finaliza com nível muito elementar de conhecimento e interação com dispositivos desktop.

\subsubsubsection{Persona 1}
José é um senhor de 65 anos que sempre trabalhou em escritórios, mas sempre contou com o suporte da empresa em que trabalhava para conseguir resolver problemas cotidianos. Entretanto, desde que se aposentou, não conta mais com esse suporte. Não é um problema muito grande, mas nem sempre existe alguém disponível para ajuda-lo e o problema acaba esquecido. Seu computador novo só é usado, quando está no cabo.

\subsubsubsection{Persona 2}
Eleonora é uma enfermeira que usa apenas os sistemas do hospital em que trabalha. Em casa, preciso instalar o sistema java no seu computador para fazer a declaração do imposto de renda do ano em seu computador novo, mas morre de fazer algo e ter dados vazados. Não sabe pra quem pedir, por que está sempre muito ocupada.

\subsubsubsection{Persona 3}
Rafael é pai de duas crianças pequenas e trabalha o dia inteiro na rua, mas precisa emitir nota fiscal e fazer algumas coisas no computador de tempos em tempos. Entretanto vira e mexe, tem que resolver problemas com o computador ou com a impressora e isso sempre toma tempo demais, tomando do tempo que gostaria de brincar com os filhos.

\subsubsubsection{Persona 4}
Claudenice faz a melhor coxinha da região e é muito organizada. Os negócios estão indo muito bem e até já contratou uma contabilidade e duas pessoas para ajudar a organizar nas atividades do dia a dia. Mas, nenhuma das pessoas que contratou pode fazer algo que ela não abre mão: tomar conta das próprias finanças. Mesmo assim, vão ajuda-la a organizar as entregas e coletar os pedidos no computador. Ninguém sabe mexer muito bem no computador e isso preocupa a Claudenice, pois já ficou na mão uma vez e demorou bastante para conseguir achar alguém que ajudasse.

\section{Objetivos}

\subsection{}

%\lipsum[3-5]
%Teste de citação para gerar referências no modelo.. \citeauthor{SCRUMGUIDE:2013}


% ----------------------------------------------------------
% Referências bibliográficas
% ----------------------------------------------------------
\bibliography{referencias,exemplos/abntex2-doc-abnt-6023}

\end{document}