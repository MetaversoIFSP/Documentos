
% ----------------------------------------------------------
% Introdução
% ----------------------------------------------------------
\chapter[Introdução]{Introdução}

De acordo com a PNAD de 2019 \citeauthor{PNAD:2019}, 82,7\% dos domicílios brasileiros contam com acesso à internet. Mesmo tendo acesso de banda larga em 80,2\% dos dispositivos móveis do país, apenas 45,1\% os domicílios da amostra têm acesso via computador. Assim, entendemos que embora muitos dos usuários têm alguma fluência em aplicativos móveis, as diferenças de interfaces pode ser um desafio para o usuário que têm tarefas diferentes das desempenhadas em aplicativos móveis.

Ainda, considerando que existem diversas soluções corporativas de suporte, em contraste, para esses usuários, as soluções que mais se apresentam são fóruns, que ainda que sejam gratuitas, ainda demandam algum trabalho e interação que os usuários alvo são pouco familiarizados ou têm compreensão limitados em relação à dinâmica de tais ferramentas.

Assim, propomos uma ferramenta com capacidade de suportar os usuários que sentem necessidade de suporte personalizado.

\section{Questão de Pesquisa}

Existência de potencial relevante de usuários de pouca fluência com soluções de tecnologia que têm dificuldades em resolver problemas frustantes e corriqueiros com computadores pessoais que  incomodam ou impedem o uso esperado do dispositivo por não conseguirem desfrutar do dispositivo ou, mesmo, trabalhar.



O projeto elaborado propõe uma solução que atenda pessoas com baixa fluência em sistemas de computação em situações cotidianas em que seus dispositivos não funcionem de acordo com o esperado pelo usuário.

Entende-se que a solução tem como alvo pessoas físicas e pequenas empresas, que normalmente têm acesso limitado ou nenhum a ferramentas tradicionais de suporte de tecnologia da informação (\textit{ITSM - Information Technology Service Management}).

Para tanto, foram definidos os seguintes objetivos para criação do serviço: elaboração do mínimo produto viável e suas ferramentas essenciais, com pontos de checagem (\textit{check point}) para verificação do avanço da solução.

\section{Objetivo Principal}

Disponibilizar um serviço de assistência a problemas em sistemas computacionais domésticos e pequenas empresas para usuários finais com pouca ou nenhuma familiaridade a problemas cotidianos.

\section{Objetivos Secundários}

Para que o produto de ITSM - \textit{Information Technology Service Management} seja considerado viável, será necessário que as seguintes ferramentas sejam disponibilizadas as funcionalidades a seguir: 

\subsection{Ferramenta de respostas rápidas para perguntas frequentes: FAQ - \textit{Frequently Asked Questions}}

Para a solução de FAQ, compreendeu-se a necessidade de desenvolvimento das ferramentas:

\begin{enumerate}
	\item Modelo de árvore de decisão
	
	Realização de consultar em banco de dados relacional baseado em SQL - \textit{Server Query Language}, o qual localiza uma resposta de solução ao problema alegado.
	
	\item Tela inicial
	
	Implementação de ferramenta de busco para agregar na consulta otimizada afim de facilitar a procura de problemas relacionados.
	
	\item Adicionar Cookie de sessão para armazenar o comportamento do usuário. (verificar IP - \textit{Internet Protocol} - público e contexto LGPD - Lei Geral de Proteção de Dados)
	
	Armazenamento do IP público do usuário para identificar o seu comportamento no site, seguindo critérios de aceite aos termos e políticas condicionais no site, baseadas na LGPD.
	
	\item Cadastro no FAQ
	
	O cadastro no FAQ é realizado pelo próprio time da central de suporte técnico, quando identificam um novo problema que está sendo relatado com muita frequência, busca uma solução otimizada e disponibiliza no FAQ. 
\end{enumerate}

\subsection{Login}

\begin{enumerate}
	\item Login confiável por autorização de acesso
\end{enumerate}

\subsection{Cadastros}

\begin{enumerate}
	\item Criação de cadastro de perfil
	
	Após a realização do cadastro com o mínimo necessários de informação o cliente poderá fazer um preenchimento complementar do seu perfil, demonstrando:
	
	\begin{enumerate}
		\item Marcas e modelos de seus equipamentos;
		\item Quantidade de usuários e nome dos usuários no local;
		\item Software de que gosta de utilizar;
		\item Outras opções.
	\end{enumerate}
	
	\item Criação de cadastro de perfil técnico
	
	Após a realização do cadastro com o mínimo necessários de informação o cliente poderá fazer um preenchimento complementar do seu perfil, demonstrando
	
	\begin{enumerate}
		
		\item Marcas e modelos de seus equipamentos que atende;
		\item Formação profissional;
		\item Especialização;
		\item Área que realizará o atendimento de preferência;
		\item Outras opções. 
		
	\end{enumerate}
	
	\item Página de Cursos e Capacitação
	
	Os técnicos terão acesso a cursos que poderão realizar na plataforma afim de aprimorar o atendimento e capacitação e precisaram realizar uma avaliação técnica básica para realizar o atendimento.
	
\end{enumerate}

\subsection{Visualização do perfil e escolha personalizada do técnico}

\begin{enumerate}
	
	\item Visualização de cadastro de perfil
	
	O cliente poderá receber o perfil do técnico e sua média de avaliação e suas especialidades.
	
	\item Escolha de técnicos personalizada por critérios de avaliação
	
	O cliente poderá por meio de um plano especifico contratar um técnico com um perfil adequado a sua necessidade e baseado em sua avaliação.
	
\end{enumerate}

\subsection{Agendamento de visita técnica}

O cliente poderá por meio de um plano especifico contratar um técnico com um perfil adequado a sua necessidade e baseado em sua avaliação.

\begin{itemize}
	
	\item Ter opção de botão dedicado para o usuário final abrir um chamado de visita técnica on site.
	
	Opção de chamado técnico facilitado, o cliente com um cadastro simples, sem necessidade de acessar o FAQ poderá solicitar um técnico até o local, sendo guiado pelo processo de agendamento.
	
	\item Agendamento por meio de raio de localidade
	
	Durante o processo de agendamento a escolha do técnico e feita é realizado por localidade do técnico registrado em uma determinada região.
	
\end{itemize}

\subsection{Operacional}

\begin{enumerate}
	
	\item Abertura do chamado técnico
	
	Todos os atendimentos técnicos realizados pela central de atendimento ou diretamente na visita técnica vão gerar uma abertura de um chamado técnico (incidente).
	
	\item Envio de foto do problema do equipamento
	
	O chamado possui campo para adicionar fotos do problema técnico alegado pelo cliente para armazenamento de histórico.
	
\end{enumerate}

\subsection{Planos de Compra}

\begin{enumerate}
	
	\item Free
	
	Usuário acessa a plataforma, navega pelos FAQs, podendo sanar suas dúvidas e resolver seus problemas por conta própria através da plataforma – Contém ADS.
	
	\item Basic
	
	Modelo de assinatura mensal onde o usuário paga um valor e terá 1 dispositivo vinculado à assinatura. Ao assinar, a equipe instalará os softwares para acesso remoto e, quando o usuário não conseguir resolver por conta própria o problema, será atendido via chat ou WhatsApp para resolução. Para atendimento técnico, os técnicos são escolhidos de forma aleatória dando preferência a região.
	
	\item Premium
	
	Modelo de assinatura mensal onde o usuário paga um valor e terá 5 dispositivos vinculado à assinatura. Ao assinar, a equipe instalará os softwares para acesso remoto e, quando o usuário não conseguir resolver por conta própria o problema, será atendido via chat ou WhatsApp. Para atendimento técnico, os técnicos são escolhidos de forma aleatória dando preferência a região e escolha de técnicos mais bem avaliados.
	
\end{enumerate}

\subsection{Avaliação da visita técnica.}

\begin{enumerate}
	
	\item Botão de Resolução (Sim ou Não)
	
	Após a conclusão da visita técnica aparecerá para o cliente uma pergunta se o problema foi resolvido com dois botões (Sim e Não), caso tenha resolvido apresentará a mensagem de dúvidas, sugestões ou reclamações. Em caso de não solução o incidente voltará a ser reportado para o time de atendimento técnico analisar o caso. 
	
	\item Avaliação da visita técnica
	
	Funcionalidade de avaliação da visita técnica do cliente com critérios de nota:(1 muito pouco satisfeito, 2 pouco satisfeito, 3 regular, 4 satisfeito, 5 muito satisfeito), o qual os técnicos mais bem avaliados terão mais chances de receber um contato de visita técnica. Existindo campos também para sugestões e reclamações.
	
\end{enumerate}

\subsection{Modo de visualização}

\begin{enumerate}
	
	\item Modo de aplicação em WEB
	
	O serviço será disponibilizado em formato de aplicação WEB acessível em qualquer navegador com interação facilitada.
	
	\item Avaliação da visita técnica.
	
	O serviço será disponibilizado em formato de aplicação WEB acessível em qualquer navegador com interação facilitada.
	
\end{enumerate}

\subsection{Avaliação de visita técnica e clientes}

\begin{enumerate}
	
	\item Avaliação da visita técnica.
	
	Funcionalidade de avaliação da visita técnica do cliente com critérios de nota:(1 muito pouco satisfeito, 2 pouco satisfeito, 3 regular, 4 satisfeito, 5 muito satisfeito), o qual os técnicos mais bem avaliados terão mais chances de receber um contato de visita técnica. Existindo campos também para sugestões e reclamações.
	
	
	\item Avaliação da visita técnica.
	
	Funcionalidade de avaliação da visita técnica do cliente com critérios de nota:(1 muito pouco satisfeito, 2 pouco satisfeito, 3 regular, 4 satisfeito, 5 muito satisfeito), o qual os técnicos mais bem avaliados terão mais chances de receber um contato de visita técnica. Existindo campos também para sugestões e reclamações.
	
	\item Avaliação do cliente.
	
	Funcionalidade de avaliação da visita técnica do cliente com critérios de nota:(1 muito pouco satisfeito, 2 pouco satisfeito, 3 regular, 4 satisfeito, 5 muito satisfeito), opção de classificar o nível de conhecimento técnico (1 nenhum conhecimento, 2 poucos conhecimento, 3 regular, 4 bons conhecimentos, 5 conhecimentos avançados).
	
\end{enumerate}

\subsection{Relatórios}

\begin{enumerate}
	
	\item Relatório de Incidentes (Classificação, gerencial, traçar perfil de recorrência de incidente e comportamento).
	
	Geração de relatório de incidentes de nível gerencial, com demonstração de informações de classificação, recorrência de incidentes e comportamento dos usuários do FAQ.
	
	\item Painel de indicadores
	
	Criação de painel de indicadores de consultas e comportamentos mais frequentes para melhoria da solução.
	
\end{enumerate}

\section{Justificativa}
% \preencheComTexto

\section{Estrutura do Estudo}
% \preencheComTexto

