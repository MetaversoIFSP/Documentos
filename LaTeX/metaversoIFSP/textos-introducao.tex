
% ----------------------------------------------------------
% Introdução
% ----------------------------------------------------------
\chapter[Introdução]{Introdução}

	De acordo com a PNAD de 2019 \citeauthor{PNAD:2019}, 82,7\% (oitenta e dois vírgula sete) dos domicílios brasileiros contam com acesso à internet, mas mesmo tendo acesso de banda larga em 80,2\% (oitenta vírgula dois) dos dispositivos móveis do país, apenas 45,1\% (quarenta e cinco vírgula um) dos domicílios da amostra têm acesso via computador. Assim, entendemos que embora muitos dos usuários tenham alguma fluência em aplicativos móveis, as diferenças de interface podem ser um desafio para o usuário que têm tarefas diferentes das desempenhadas em aplicativos móveis.
	
	Ainda, considerando que existem diversas soluções corporativas de suporte, em contraste, para esses usuários, as soluções que mais se apresentam são fóruns, que ainda que sejam gratuitas, demandam algum trabalho e interação dos quais os usuários alvos tendem a ser pouco familiarizados ou têm compreensão limitada em relação à dinâmica de tais ferramentas.
		
	O projeto elaborado propõe uma solução que atenda pessoas que possuem baixa familiaridade com sistemas computacionais de uso cotidiano, em que seus dispositivos não funcionem de acordo com a expectativa de uso normal pelo usuário, tendo a capacidade de apoiar esse público que sente necessidade de suporte personalizado por não se sentirem capazes de resolver situações cotidianas.

	\section[Questão de Pesquisa]{Questão de Pesquisa}
	
		O acesso a recursos computacionais têm se tornado cada vez mais acessível conforme as tecnologias de comunicação, tanto de software quanto de hardware, se tornam mais baratas e as velocidades continuam crescendo. Entretanto, a acessibilidade a tais recursos não significa que os indivíduos que adquirirem esses recursos, como micro computadores e celulares inteligentes, tenham plena capacidade de resolver problemas em seus dispositivos, que poderiam ser solucionadas com pesquisas simples em sites de busca.
		
		Verificamos que existem muitos usuários que têm alta dependência das facilidades e recursos disponibilizados pela democratização dos dispositivos, seja por tecnologias móveis ou desktop. Mas, a questão de facilidade é limitada pela familiaridade desses indivíduos com a resolução de situações que demandam entendimento mínimo das instruções apresentadas pelos dispositivos para possam resolver situações que muitas vezes são identificáveis em buscas simples na internet.
		
		Tais dificuldades em resolver problemas ocasionam frustanções corriqueiras desses usuários com computadores pessoais e dispositivos móveis que incomodam ou impedem o uso do dispositivo. Em situações como a de pequenos autônomos, os indivíduos podem ficar até mesmo impossibilitados de desenvolver atividades laborais.
	
	\section[Objetivo Principal]{Objetivo Principal}
	
		Disponibilizar um serviço de assistência a problemas em sistemas computacionais domésticos e pequenas empresas para usuários finais com pouca ou nenhuma familiaridade a problemas cotidianos.
	
	\section[Objetivos Secundários]{Objetivos Secundários}
	
		Para que o produto de ITSM - \textit{Information Technology Service Management} seja considerado viável, será necessário que as seguintes ferramentas sejam disponibilizadas as funcionalidades a seguir: 
	
		\begin{itemize}
			
			\item
			Familiarizar o usuário com pouca habilidade para resolver problemas de soluções simples com computadores pra que possam ser independentes;
			
			\item
			Facilitar a identificação e aplicação de resolução a problemas com computadores;
			
			\item
			Orientar os gestores da ferramenta sobre as questões e dificuldades mais comuns;
			
			\item 
			Facilitar o processo de resolução em caso de atendimento por terceiro credenciado
			
		\end{itemize}
	
	\section[Justificativa]{Justificativa}
	% \preencheComTexto
	
			
		Entende-se que a solução tem como alvo pessoas físicas e pequenas empresas, que normalmente têm acesso limitado ou nenhum a ferramentas tradicionais de suporte de tecnologia da informação (\textit{ITSM - Information Technology Service Management}).
			
		Para tanto, foram definidos os seguintes objetivos para criação do serviço: elaboração do mínimo produto viável e suas ferramentas essenciais, com pontos de checagem (\textit{check point}) para verificação do avanço da solução.
		
		A solução Metaverso é uma solução gratuita que facilita a identificação e resolução de problemas de usuários por conta própria, ao mesmo tempo que ensina seus usuários como resolver suas dificuldades imediatas fortalece o vínculo comercial para resolver situações mais complexas através de uma ferramenta proprietária de chamados.
	
%	\section[Estrutura do Estudo]{Estrutura do Estudo}
	% \preencheComTexto

