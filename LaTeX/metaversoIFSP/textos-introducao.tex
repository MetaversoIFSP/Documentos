
% ----------------------------------------------------------
% Introdução
% ----------------------------------------------------------
\chapter[Introdução]{Introdução}

	De acordo com a PNAD de 2019 \citeauthor{PNAD:2019}, 82,7\% dos domicílios brasileiros contam com acesso à internet. Mesmo tendo acesso de banda larga em 80,2\% dos dispositivos móveis do país, apenas 45,1\% os domicílios da amostra têm acesso via computador. Assim, entendemos que embora muitos dos usuários têm alguma fluência em aplicativos móveis, as diferenças de interfaces pode ser um desafio para o usuário que têm tarefas diferentes das desempenhadas em aplicativos móveis.
	
	Ainda, considerando que existem diversas soluções corporativas de suporte, em contraste, para esses usuários, as soluções que mais se apresentam são fóruns, que ainda que sejam gratuitas, ainda demandam algum trabalho e interação que os usuários alvo são pouco familiarizados ou têm compreensão limitados em relação à dinâmica de tais ferramentas.
	
	Assim, propomos uma ferramenta com capacidade de suportar os usuários que sentem necessidade de suporte personalizado.

	\section[Questão de Pesquisa]{Questão de Pesquisa}
	
		Existência de potencial relevante de usuários de pouca fluência com soluções de tecnologia que têm dificuldades em resolver problemas frustantes e corriqueiros com computadores pessoais que  incomodam ou impedem o uso esperado do dispositivo por não conseguirem desfrutar do dispositivo ou, mesmo, trabalhar.
		
		O projeto elaborado propõe uma solução que atenda pessoas com baixa fluência em sistemas de computação em situações cotidianas em que seus dispositivos não funcionem de acordo com o esperado pelo usuário.
			
		Entende-se que a solução tem como alvo pessoas físicas e pequenas empresas, que normalmente têm acesso limitado ou nenhum a ferramentas tradicionais de suporte de tecnologia da informação (\textit{ITSM - Information Technology Service Management}).
		
		Para tanto, foram definidos os seguintes objetivos para criação do serviço: elaboração do mínimo produto viável e suas ferramentas essenciais, com pontos de checagem (\textit{check point}) para verificação do avanço da solução.
	
	\section[Objetivo Principal]{Objetivo Principal}
	
		Disponibilizar um serviço de assistência a problemas em sistemas computacionais domésticos e pequenas empresas para usuários finais com pouca ou nenhuma familiaridade a problemas cotidianos.
	
	\section[Objetivos Secundários]{Objetivos Secundários}
	
		Para que o produto de ITSM - \textit{Information Technology Service Management} seja considerado viável, será necessário que as seguintes ferramentas sejam disponibilizadas as funcionalidades a seguir: 
	
		\begin{itemize}
			
			\item
			Familiarizar o usuário com pouca habilidade para resolver problemas de soluções simples com computadores pra que possam ser independentes;
			
			\item
			Facilitar a identificação e aplicação de resolução a problemas com computadores;
			
			\item
			Orientar os gestores da ferramenta sobre as questões e dificuldades mais comuns;
			
			\item 
			Facilitar o processo de resolução em caso de atendimento por terceiro credenciado
			
		\end{itemize}
	
	\section[Justificativa]{Justificativa}
	% \preencheComTexto
	
		A utilização de recursos computacionais têm sido disponibilizados de forma acessível a mais e mais pessoas. Entretanto, a acessibilidade a tais recursos não significa que os indivíduos que adquirirem esses recursos como micro computadores e celulares inteligentes têm plenas capacidades de resolver problemas em seus dispositivos, mas que poderiam ser solucionadas com pesquisas simples em sites de busca.
		
		A solução Metaverso é uma gratuita que facilite a identificação e resolução de problemas de usuários por conta própria ao mesmo tempo que ensine aos seus usuários como resolver suas dificuldades imediatas fortalece o vínculo comercial para resolver situações mais complexas através de uma ferramenta proprietária de chamados.
	
%	\section[Estrutura do Estudo]{Estrutura do Estudo}
	% \preencheComTexto

