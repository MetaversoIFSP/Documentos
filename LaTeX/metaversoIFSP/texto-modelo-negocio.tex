\chapter{Modelo de negócio}

O modelo de negócio contará com planos free e escalonado em dois diferentes níveis de serviço para permitir que os clientes possam escolher o nível de suporte que deseja ser atendido.

Em ambos os modos, existe a possibilidade do usuário, após passar pelos filtros da plataforma para sua solução de problema, acionar um prestador de serviços terceiro, que poderá atender o usuário presencialmente em seu estabelecimento. Nesses casos, a plataforma poderá receber o pagamento do serviço (tabelado) e reter o mesmo até confirmação da solução por ambas as partes, funcionando como uma intermediadora.


Futuramente, poderão ser estudados planos de assinaturas premium em níveis diferenciados.

\section{Planos}
São divididos em dois planos: free e premium.

\subsection{Free}

Usuário acessa a plataforma, navega pelos FAQs, podendo sanar suas dúvidas e resolver seus problemas por conta própria através da plataforma – Contém ADS.

\subsection{Premium}

Modelo de assinatura mensal onde o usuário paga um valor e terá 1 dispositivo vinculado à assinatura. Ao assinar, a equipe instalará os softwares para acesso remoto e, quando o usuário não conseguir resolver por conta própria o problema, será atendido via chat ou WhatsApp para resolução.


\section{MVP}

A proposta inicial é a criação de uma ferramenta que favoreça a identificação da solução pelo usuário por conta própria, utilizando orientação de uma solução automatizada.

Nesse momento, o desenvolvimento de uma solução que popule e atualize a seção de perguntas frequentes e o a possibilidade de agendamento técnico é prioridade de desenvolvimento e será o principal indicador de viabilidade.
